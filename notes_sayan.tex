In \refsect{syntax}, we described the structure of applications written in this language; each application is written in terms of events. We say that an event has occurred if its precondition was evaluated as true, and the corresponding effect was executed. Once an event has occurred, the program control goes back to the top of the event block. The event block (\s{eventBlock}) is said to be executed when any event in the event block occurs once. Given an agent running an application, its event block can be executed repeatedly while the application continues to run. Communication between agents takes place through shared memory updates. Shared memory updates require mutual exclusion, which is implemented by a locking mechanism. We describe the \verb|atomic| construct, which is used for mutual exclusion, in \refsect{locking}.
