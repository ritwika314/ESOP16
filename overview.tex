\subsection{Overview}
\label{sect:Overview}
\rolang allows users to write applications that will run on a distributed system of agents. The user writes a program as though for a single agent, and all agents execute the same code. A \rolang program is a collection of \emph{declarations} and \emph{events}. Shared variables are used for communication between agents. They are declared in an \verb|MW| block. \verb|MW| stands for \emph{multi-writer}, implying that all agents can read from, and write to the variables declared in this block. We provide another type of shared variables called \emph{shared single-writer} variables, which all agents can read from, but only one agent can write to. These variables are parameterized by the \verb|agent index|, an integer which is a unique identifier for each agent in the system, and they are declared in the \verb|SW| block. Local variables are declared in \verb|Loc|al declaration blocks.  \newline

As mentioned earlier, \rolang uses a \emph{precondition-effect} style of programming. These precondition and effect statements form events. Each application consists of a special \verb|Init| event, and an \verb|EventBlock|. The \verb|Init| event occurs at the beginning when the application starts executing, and it contains all statements which need to be executed only once; for instance, initializing a shared array. After the \verb|Init| event is executed, the \verb|EventBlock| starts executing. It contains a list of events that define the behavior of the application. The preconditions of each of the events inside the \verb|EventBlock| are evaluated in order of appearance, and the effect is executed if the precondition becomes true. The \verb|EventBlock| can be seen as a (potentially) infinite while-loop. 

We also provide an abstraction to manage the physical control of the agents, called \verb|doReach|, which takes as input a \emph{target} to reach, and a list of (predetermined) \emph{obstacles} which need to be avoided. We do not need to specify the format of the obstacles, as different implementations of this \verb|doReach| can have different specifications, but the target in general has the same type as the time varying variables of the system. \verb|doReach| communicates with the program using two flags, \emph{doReach_done}, and \emph{doReach_fail}. If the target is reached, the \emph{doReach_done} is set to true, and the \emph{doReach_fail} is set to true when the agent does not seem to have reached the target.

The next section presents the formal syntax, and an example to illustrate the structure of a general application. 

\subsection{Syntax}
\label{sect:syntax}
This section describes the formal syntax of \rolang. We first provide the major features of the formal syntax which describe program structure, event structure, and statement structure. 
\begin{figure}[ht!]
\footnotesize
\begin{center}
\noindent\begin{minipage}{.5\textwidth}
\begin{grammar}
<Pgm> :: <VarDecls> <InitBlock> <EventBlock>

<VarDecls> :: <MwDecls> <SwDecls> <LocDecls>

<MwDecls> :: "MW :" <Decls>

<SwDecls> :: "SW :" <Decls>

<LocDecls> :: "Loc :" <Decls>

<Decls> :: <Decl> <Decls> \\
         | <Empty>

<Decl> :: <EnumDecl> ";" \\
         | <ArrayDecl> ";" \\
         | <Type> <Var> ";"\\
         | <Type> <Var> "=" <Expr>
 
<InitBlock> :: "Init :" <Stmts>
\end{grammar}
\end{minipage}\hfill
\noindent\begin{minipage}{.5\textwidth}
\begin{grammar}
<EventBlock> :: "EventBlock:" <Events>

<Events> :: <Event> <Events> | <Event>

<Event> :: <EventName> "(" <Expr> ")" "Pre" "(" <Expr>")" ";" "Eff :" <Stmts> 

<Stmts> :: <Stmt> <Stmts> \\
		| <Empty>

<Stmt> :: <Assignment> \\
		| <If-Then-Else> \\
        | <Loop>\\
        | <Atomic>
		| <FunctionCall>
        
<Atomic> :: "Atomic :" Stmts
\end{grammar}
\end{minipage}
\end{center}
\caption{Language Syntax Features}
\end{figure}
As mentioned in the overview, each program consists of three major parts, with variable declarations, an initialization block and an event block. Aside from usual data types and arrays, we provide support for declaring enumerated types, as it is easy to use them as "stages" in applications.\footnote{https://github.com/ritwika314/RoLang/StarL/HLL/Examples/LeaderElect}

Events, as mentioned earlier are specified by precondition-effect blocks, where the precondition is a boolean expression, and effect blocks contain statements, which can be assignment statements, \verb|if-then-else| statements, \verb|Atomic| statements, function calls, or loops. We omit the productions for the more obvious syntactic elements like expressions, assignment statements, loops, etc.\footnote{https://github.com/ritwika314/RoLang/StarL/HLL/Semantics/agent-syntax.k}
