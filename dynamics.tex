\subsection{Dynamics}
\label{sect:dynamics}
We first present the semantic rules involving the \s{doReach} abstraction. The physical environment uses the \s{doReach} component to communicate with the application, which in turn uses flags \s{doReach\_done} and \s{doReach\_fail} to store the continuous control variables used by the application. In this case, \s{position} is such a variable. Recall that \s{doReach} takes two arguments, the \emph{target}, and the \emph{obstacle}. The exact format of the obstacle is irrelevant to the semantics, as it is implementaion specific to the \s{doReach} external function. This function is invoked whenever time advances. The target and the obstacles are set to current position and empty initially, unless the application specifies otherwise. 

\paragraph{doReach invoke transition} The \s{transitionState} cell of the system is set to \s{continuous} when all the \s{counter} values of all agents reach $n_0$. Suppose that the last observed position was $P$, and the time it was observed at was $T_0$, the target is $T$, and the obstacles are stored in $O$. Time increments by $\delta$ during an environment transition. These variables are also updated at the end of every round, but we omit those details. 
%

\begin{figure}
\label{fig:doreach2}
\krule[doreach invoke transition]{
\kprefix{agent}{
\begin{array}{@{}c@{}}
\kall{id}{\variable[Int]{i}}
\kall{counter}{\reduce{N}{0}}
\kall{position}{\reduce{\variable[ItemPosition]P}{P_{1}}}
\end{array}
}\mathrel{}
\kmiddle{counterMap}{\reduce{i\mapsto N}{i \mapsto 0}}
\kall{transitionState}{continuous}}
{N \mathrel{==_{\scriptstyle\it Int}} n_0 \terminal{and} P_{1} = \terminal{doReachBB}(P,T,O,T_0,\delta,i)}{}{}{}
\end{figure}



The rule to process a \s{doReach} statement  in the application is as follows. 
\paragraph{doReach update transition: }When a \s{doReach} statement is encountered in the program, the statement itself is rewritten to empty. The $\curvearrowright$ is used to break down the program, which is seen as a single task, into a sequence of tasks. Therefore, it means after rewriting the \s{doReach} to empty, the \s{doReach} flags should be reset(\s{doReach\_done,doReach\_fail} are both set to false), which should in turn be followed by setting the target of this agent, and the list of obstacles. The \s{setTargetObs} function sets the values of target, obstacle, and current time. The external function \s{doReachBB} is called on these values at the end of this round of program transitions.  
\begin{figure}
\label{fig:doreach1}
\krule[doreach update transition]{
\kprefix{agent}{
\begin{array}{@{}c@{}}
\kprefix{k}{
 \reduce{\terminal{doReach} (T,O)}{{\dotCt{}}} \curvearrowright resetFlags(i) \curvearrowright setTargetObs(T,O,T_0,i)
} 
\kall{id}{\variable[Int]{i}}\mathrel{}
\end{array}
}
\kall{time}{T_0}
}{}{}{}{}
\end{figure}
