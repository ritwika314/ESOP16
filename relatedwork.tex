\subsection{Related Work}
\label{sect:rel}
In this work, we have presented the formal semantics of a  language for distributed agent coordination and control. The main focus of this work was on developing a formal model for asynchronous concurrent applications where communication occurs through shared memory. \cite{P} is a language for asynchronous event-driven programming, which allows the programmer to specify the system as a collection of interacting state machines, which communicate with each other using events as opposed to shared memory updates as in \rolang. In an actual implementation (like StarL), agents do respond to message events, which could in principle be modeled in P. Our work provides  a framework that allows a relatively novice user to write pseudocode without being concerned with implementation details.
There are languages such as Esterel \cite{esterel} , Lustre \cite{lustre} and Signal\cite{signal}. As in our model of time evolution, these languages also follow a model where time
advances in steps. However, since we express our semantics in the \K framework, we can explore various interleaving semantics. In these languages, given
a state and an input at the current time step, there is a unique possible
state at the next time step. 
The synchronous model has the advantage that every event
sent to machine is handled in the next clock tick, and is widely
used in hardware and embedded systems. However, in an OS or a
distributed system, it is impossible to have all the components of
the system clocked using a global clock, and hence asynchronous
models are used for these systems, which gives a language like P an advantage over \rolang. In such models events are
queued, and hence can be delayed arbitrarily before being handled. Theoretically, we can also model delays like this by enforcing application of specific rewrite rules repeatedly, but that would limit the generality of the semantics. 
