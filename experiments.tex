\section{Experiments}
We show two of the case studies we performed using the executable semantics we just defined. 
\subsection{Fischer's Protocol}
Agents try to access a critical section mutually exclusively. Each agent is defined as follows. The agents have a shared variable called \s{reqid} which is used to request entry into the critical section. Another shared variable \s{k} is used to define wait times and entry times. Each agent initially checks waits for a time between 0 and \s{k}, (tracked by \s{c1}, and if the \s{reqid} is not set (\s{reqid} is $-1$), then it sets \s{reqid} atomically to its own \s{id}. It then waits for \s{d} time (tracked by \s{c2}), and checks whether the \s{reqid} is its own \s{id}. If that is the case, then it enters the critical section, otherwise it goes back to waiting. If a robot enters the critical section, we make it spend \s{cstime}(tracked by \s{c3} units of time in the critical section to help detect mutual exclusion violations more easily.

Since the passage of time should correspond to the increments of the variables \s{c1},\s{c2} and \s{c3}, we set $n_0$ to 1, so that the \s{time} increments every time the \s{counter} of all agents increments.  
\begin{figure}[ht!]
\label{fig:fish}
\noindent\begin{minipage}{.5\textwidth}

\begin{lstlisting}
Agent::Fischer

allwrite:
  int reqid = -1; 
  int k = 10;
  int d = 2;
allread:

loc:
  int id = getAgentIndex();
  int waitTime = rand(k);
  bool start = true;
  bool assign = false;
  bool delay = false;
  bool wait = false;
  
  bool inCs = false;
  int c1 = 0;
  int c2 = 0;
  int cstime = rand(5);
  int c3 = 0;
  
Init:

Start():
  pre(start);
  eff:
  	if (reqid != -1):
    	c1 = 0;
    else:
      start = false;
      wait = true;

Wait():
  pre(wait);
  eff:
      if(c1 < waitTime):
          c1 = c1+1;
 \end{lstlisting}
 \end{minipage}\hfill
\noindent\begin{minipage}{.5\textwidth}

\begin{lstlisting}  
      else:
      	c1 = 0;
        wait = false;
        assign = true;

Assign():
  pre(assign);
  eff:
    atomic:
   	 reqid = id;
    delay = true;
    assign = false;
    
Delay():
  pre(delay);
  eff:
    if (c2 < d):
    	c2 = c2+1;
    else:
      if (reqid != id):
        c2 = 0;
        c1 = 0;
        delay = false;
        wait = true;
      else:
      	inCs = true;
        wait = false;

InCs():
	pre(inCs);
    eff: 
    	if (c3 < 3):
        	c3 = c3 + 1 ;
        else:
          c3 = 0;
          inCs = false;
          start = true;
          atomic:
            reqid = -1;
    	
 \end{lstlisting}
 \end{minipage}\hfill
 \caption{Fischer's  protocol}
 \end{figure}
 
 To ensure that this protocol works, the local \s{inCs} variable of at most one agent can be true at \s{time} $T$. The assignment stage takes at least 2 increments of time, as we increment the counter every time an agent makes a lock request, and in this example, the time also increments as soon as the counter is incremented. The following execution trace shows that for \s{d} = 2, we can violate this property. For bounded executions, we were unable to find traces which violated this property when \s{d} was set to a value greater than 2 , which is expected. 

\begin{verbatim}
 <agent> ... <memory> start |-> 8 wait |-> 11 inCs |-> 12 cstime |-> 15 
 c1 |-> 13 c3 |-> 16 reqid |-> 3 id |-> 6 k |-> 4 waitTime |-> 7 assign |-> 9
 delay |-> 10 d |-> 5 c2 |-> 14 
 </memory> <id> 0 </id> 
 </agent> 
 <agent> ... <memory> start |-> 22 wait |-> 25 inCs |-> 26 cstime |-> 29
 c1 |-> 27 c3 |-> 30 reqid |-> 17 id |-> 20 k |-> 18 waitTime |-> 21 
 assign |-> 23 delay |-> 24 d |-> 19 c2 |-> 28 
 </memory> <id> 1 </id> 
 </agent>
 <gMemory> reqid |-> 0 k |-> 1 d |->  2 </gMemory>
 <MMap> 0 |-> -1 1 |-> 10 2 |-> 2 3 |-> -1 4 |-> 10 5 |-> 2 6 |-> 0 7 |-> 6
 8 |-> true 9 |-> false 10 |-> false 11 |-> false 12 |-> false 13 |-> 0
 14 |-> 0  15 |-> 3 16 |-> 0 17 |-> -1 18 |-> 10 19 |-> 2 20 |-> 1 
 21 |-> 3 22 |-> true 23 |-> false 24 |-> false 25 |-> false 26 |-> false
 27 |-> 0 28 |-> 0 29 |-> 5 30 |-> 0 
 </MMap>
 <time> 0 </time>
\end{verbatim}


At time = 0; both the agents have executed the \s{Start} event, indicated by the fact that the \s{start} variables of both agents, corresponding to addresses \verb|8| and \verb|22| respectively, map to \verb|false| in the \s{MMap}. The \s{waitTime} for agent with agent 0 is stored at address \verb|7|, and is seen from the \s{MMap} to be 6. The \s{waitTime} for agent 1 is seen to be 3.   

At time = 4, agent 1 is seen to be executing \s{Assign} event, as evidenced by the fact that variable \s{assign} for agent 1 stored at address \verb|23| is true. Agent 0 is still waiting to start its \s{Assign} event, since its \s{c1} is still 4 and its \s{waitTime} is 6.  
\begin{verbatim}
 <agent> ... <memory> start |-> 8 wait |-> 11 inCs |-> 12 cstime |-> 15 
 c1 |-> 13 c3 |-> 16 reqid |-> 3 id |-> 6 k |-> 4 waitTime |-> 7 assign |-> 9
 delay |-> 10 d |-> 5 c2 |-> 14 
 </memory> <id> 0 </id> 
 </agent> 
 <agent> ... <memory> start |-> 22 wait |-> 25 inCs |-> 26 cstime |-> 29
 c1 |-> 27 c3 |-> 30 reqid |-> 17 id |-> 20 k |-> 18 waitTime |-> 21 
 assign |-> 23 delay |-> 24 d |-> 19 c2 |-> 28 
 </memory> <id> 1 </id> 
 </agent>
 <gMemory> reqid |-> 0 k |-> 1 d |->  2 </gMemory>
 <MMap> 0 |-> -1 1 |-> 10 2 |-> 2 3 |-> -1 4 |-> 10 5 |-> 2 6 |-> 0 7 |-> 6
 8 |-> false 9 |-> false 10 |-> false 11 |-> true 12 |-> false 13 |-> 4
 14 |-> 0  15 |-> 3 16 |-> 0 17 |-> -1 18 |-> 10 19 |-> 2 20 |-> 1 
 21 |-> 3 22 |-> false 23 |-> true 24 |-> false 25 |-> false 26 |-> false
 27 |-> 0 28 |-> 0 29 |-> 5 30 |-> 0 
 </MMap>
 <time> 4 </time>
\end{verbatim}


At time = 11, we see that the \s{inCs} values of both agents are true, which violates the correctness of this protocol. 

 
\begin{verbatim}
 <agent> ... <memory> start |-> 8 wait |-> 11 inCs |-> 12 cstime |-> 15 
 c1 |-> 13 c3 |-> 16 reqid |-> 3 id |-> 6 k |-> 4 waitTime |-> 7 assign |-> 9
 delay |-> 10 d |-> 5 c2 |-> 14 
 </memory> <id> 0 </id> 
 </agent> 
 <agent> ... <memory> start |-> 22 wait |-> 25 inCs |-> 26 cstime |-> 29
 c1 |-> 27 c3 |-> 30 reqid |-> 17 id |-> 20 k |-> 18 waitTime |-> 21 
 assign |-> 23 delay |-> 24 d |-> 19 c2 |-> 28 
 </memory> <id> 1 </id> 
 </agent>
 <gMemory> reqid |-> 0 k |-> 1 d |->  2 </gMemory>
 <MMap> 0 |-> 0 1 |-> 10 2 |-> 2 3 |-> 0 4 |-> 10 5 |-> 2 6 |-> 0 7 |-> 6
 8 |-> false 9 |-> false 10 |-> false 11 |-> false 12 |-> true 13 |-> 0
 14 |-> 0  15 |-> 3 16 |-> 0 17 |-> 0 18 |-> 10 19 |-> 2 20 |-> 1 
 21 |-> 3 22 |-> false 23 |-> false 24 |-> false 25 |-> false 26 |-> true
 27 |-> 0 28 |-> 0 29 |-> 5 30 |-> 3 
 </MMap>
 <time> 11 </time>
\end{verbatim}

During this execution, as agent 0 was assigning the \s{reqid} to its own \s{id}, agent 1 had entered the \s{delay} state. 
When the agent 1 checked whether it can enter the critical section, agent 0 hadn't assigned \s{reqid} to its own it. Agent 1 set \s{inCs} to true, which signifies that it is in the critical section, and still is in it when agent 0 enters after satisfying all requirements. 


\subsection{Race with non-atomic check}
We revisit the race example from earlier, where a group of robots try to race to a predetermined set of points. In the \s{PickDest} state, the robots choose the next destination to race to.Then in the \s{Remove} stage, each robot \emph{atomically} updates the list of destinations to be visited if it reached the current destination. Recall that we put the check for whether an element is present in the shared list \s{dests} inside the atomic block. We now perform experiments with two versions of \s{doReachBB}, the physical control black box, where the atomic block only contains the update, but not the membership checking.  

\begin{figure}[ht!]
\label{fig:Race}
\noindent\begin{minipage}{.5\textwidth}

\begin{lstlisting}
Agent::Race

allwrite:
	List<ItemPosition> dests 
    	= getInput();
allread:

loc:
  ObstacleList obs = getObs();
  boolean Pick = true; 
  ItemPosition currentDest;
Init:

PickDest():
  pre(Pick);
  eff:
    if (isEmpty(dests)):
    exit();   
 \end{lstlisting}
 \end{minipage}\hfill
\noindent\begin{minipage}{.5\textwidth}

\begin{lstlisting}
    else:
      currentDest = head(dests);
      doReach(currentDest,obs);
      Pick = false;
      
Remove():
  pre(!Pick);
  eff:
    if(doReach_done):
      atomic:
        if(contains(
        dests,currentDest)):
        remove(dests,currentDest);
   	  Pick = true;
 \end{lstlisting}
 \end{minipage}\hfill
 \caption{Race Application}
 \end{figure}
 


\reffig{list1} shows an execution in which the agent cannot process the statement which asks to remove an \s{ItemPosition} from the \s{dests}, as between the time that it checked for membership and tried to update the list atomically, another agent already managed to remove the said \s{ItemPosition}.

\subsection{StarL robotic framework}

We are developing \rolang to interface with the Stabilizing Robotics Language (\StarL). 
\StarL is primarily in Java, and it provides language constructs for coordination and control
across robots.Two key features of \StarL are a distributed shared memory (DSM) primitive
for coordination and a reach-avoid primitive for control. DSM allows a
program to declare program variables that are shared across multiple robots
. This enables programs running on different
robots to communicate by writing-to and reading from the shared variable.

All the program threads implementing the application, the message channels,
as well as the physical environment of the application (robot chassis, obstacles)
are modeled as hybrid automata, and the overall system is described
by a giant composition of these automata. 