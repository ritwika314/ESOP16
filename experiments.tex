\section{Experiments}
We show two of the case studies we performed using the executable semantics we just defined. 
\subsection{Fischer's Protocol}
Agents try to access a critical section mutually exclusively. Each agent is defined as follows. The agents have a shared variable called \s{reqid} which is used to request entry into the critical section. Another shared variable \s{k} is used to define wait times and entry times. Each agent initially checks waits for a time between 0 and \s{k}, (tracked by \s{c1}, and if the \s{reqid} is not set (\s{reqid} is $-1$), then it sets \s{reqid} atomically to its own \s{id}. It then waits for \s{d} time (tracked by \s{c2}), and checks whether the \s{reqid} is its own \s{id}. If that is the case, then it enters the critical section, otherwise it goes back to waiting. If a robot enters the critical section, we make it spend \s{cstime}(tracked by \s{c3} units of time in the critical section to help detect mutual exclusion violations more easily.

Since the passage of time should correspond to the increments of the variables \s{c1},\s{c2} and \s{c3}, we set $n_0$ to 1, so that the \s{time} increments every time the \s{counter} of all agents increments.  
\begin{figure}[ht!]
\label{fig:fish}
\noindent\begin{minipage}{.5\textwidth}

\begin{lstlisting}
Agent::Fischer

allwrite:
	int reqid = -1; 
    int k = 5;
    int d = 3;
allread:

loc:
	int id = getAgentIndex();
	int waitTime = rand(k); 
    bool inCs = false;
    int c1 = 0;
    int c2 = 0;
	int cstime = rand(5);
    int c3 = 0;
Init:

request():
  pre(c1<waitTime);
  eff:
  	c1=c1 + 1;
wait():
  pre(c1 == waitTime and c2 < d);
  eff:
      if(reqid == -1):
          c2 = 0;
 \end{lstlisting}
 \end{minipage}\hfill
\noindent\begin{minipage}{.5\textwidth}

\begin{lstlisting}
 		atomic:
        reqid = id;
        c2 = c2+1;  
      else:
      	c2 = c2+1;

TryCs():
  pre(c2 == d);
  eff:
    if (reqid == id) :
    	inCs = true;
    c2 = 0;
    c1 = 0;

InCs():
	pre(inCs == true);
    eff:
    	if (c3 < cstime):
        	c3 = c3 +1;
        else:
        	c3 = 0;
            c1 = 0;
            inCs = false;
            atomic:
        		if (eqid == id):
                	reqid = -1
   

 \end{lstlisting}
 \end{minipage}\hfill
 \caption{Fischer's  protocol}
 \end{figure}
 To ensure that this protocol works, the local \s{inCs} variable of at most one agent can be true at \s{time} $T$. \reffig{output1} shows that for \s{k} = 5 and \s{d} = 3, we can violate this property. For bounded executions, we were unable to find traces which violated this property when \s{d} was set to a value greater than \s{k}. 
 
 %\input{RaceMod.tex}
 
 