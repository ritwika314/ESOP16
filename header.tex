\PassOptionsToPackage{pdftex,usenames,dvipsnames,svgnames,x11names}{xcolor}
\PassOptionsToPackage{pdftex}{hyperref}
\usepackage[style=math]{k}


% Slightli modified original version of \reduce. Altered baseline for more compactness.
% No support for multiline.
\newcommand{\reduceClassic}[2]{\hbox{%
  \begin{tikzpicture}[baseline=(top.south), %(top.base), - default, less compact
                      inner xsep=0pt,
                      inner ysep=.3333ex,
                      minimum width=2em]
    \path
          % Original version. No support for line wrapping.
          node (top) [inner ysep=1ex]{$#1$ \mathstrut}

          % New version. Line wrapping support.
          %node (top) [inner ysep=1ex]{$ \begin{array}{@{}c@{}} #1 \end{array} $ \mathstrut}
          (top.south)
          % Original version. No support for line wrapping.
          node (bottom) [anchor=north, inner ysep=.5ex] {$#2$};

          % New version. Line wrapping support.
          % Adds a little bit of vertical space, but the difference is truly insignificant. All the experiments below failed to remove it.
          %node (bottom) [anchor=north, inner ysep=.5ex] {$ \begin{array}{@{}c@{}} #2 \end{array} $};
          % no extra effect
          % node (bottom) [anchor=north, inner ysep=.5ex] {\vspace{-1em} $ \begin{array}{@{}c@{}} #2 \end{array} $};
          % trying mathstrut - some horizontal re-alignment, but no vertical
          % node (bottom) [anchor=north, inner ysep=.5ex] {$ \begin{array}{@{}c@{}} #2 \end{array} $ \mathstrut};
          % no outer ysep (if no inner - looks bad)
          %node (bottom) [anchor=north, inner ysep=.5ex, outer ysep=0] {$ \begin{array}{@{}c@{}} #2 \end{array} $};
          % \vskip -1em just don't compile no matter where we put it
    \path[draw,thin,solid] let \p1 = (current bounding box.west),
                               \p2 = (current bounding box.east),
                               \p3 = (top.south)
                           in (\x1,\y3) -- (\x2,\y3);
    % Solid arrow (augmenting the solid line).
    \path[fill] (top.south) ++(2pt,0) -- ++(-4pt,0) -- ++(2pt,-1.5pt) -- cycle;
  \end{tikzpicture}%
}}

% Defalut version of \reduce in this document.
%   Support for multi-line LHS and RHS
%   Good compactness. Separators adjusted to be aligned with \reduceClassic
\newcommand{\reduceMulti}[2]{\hbox{%
  \begin{tikzpicture}[baseline=(top.south), %(top.base), - default, less compact
                      inner xsep=0pt,
                      inner ysep=.3333ex,
                      minimum width=2em]
    \path
          % New version. Line wrapping support.
          node (top) [
            %inner ysep=1ex
            inner ysep=0.6ex
          ]{ $ \begin{array}{@{}c@{}}
                #1
               \end{array} $ \mathstrut}
          (top.south)
          % New version. Line wrapping support.
          node (bottom) [
            anchor=north,
            %inner ysep=.5ex
          ] {
            $ \begin{array}{@{}c@{}}
              #2
            \end{array} $};
    \path[draw,thin,solid] let \p1 = (current bounding box.west),
                               \p2 = (current bounding box.east),
                               \p3 = (top.south)
                           in (\x1,\y3) -- (\x2,\y3);
    % Solid arrow (augmenting the solid line).
    \path[fill] (top.south) ++(2pt,0) -- ++(-4pt,0) -- ++(2pt,-1.5pt) -- cycle;
  \end{tikzpicture}%
}}

%Special version of \reduce with modified baseline, for better rendering of multiline
% LHS and RHS
\newcommand{\reduceCompact}[2]{\hbox{%
  \begin{tikzpicture}[baseline=(bottom), %(top.base), - default, less compact
                      inner xsep=0pt,
                      inner ysep=.3333ex,
                      minimum width=2em]
    \path
          node (top) [inner ysep=0.6ex]{$ \begin{array}{@{}c@{}} #1 \end{array} $ \mathstrut}
          (top.south)
          node (bottom) [anchor=north] {$ \begin{array}{@{}c@{}} #2 \end{array} $};
    \path[draw,thin,solid] let \p1 = (current bounding box.west),
                               \p2 = (current bounding box.east),
                               \p3 = (top.south)
                           in (\x1,\y3) -- (\x2,\y3);
    % Solid arrow (augmenting the solid line).
    \path[fill] (top.south) ++(2pt,0) -- ++(-4pt,0) -- ++(2pt,-1.5pt) -- cycle;
  \end{tikzpicture}%
}}

%\renewcommand{\reduce}[2]{\reduceClassic{#1}{#2}}
\renewcommand{\reduce}[2]{\reduceMulti{#1}{#2}}




\lstset{language=Java,captionpos=t,tabsize=3,frame=no,keywordstyle=\color{blue},
        commentstyle=\color{gray},stringstyle=\color{red},
        breaklines=true,showstringspaces=false,emph={label},
        basicstyle=\ttfamily}

% Required in order to make \kall cells inside comments black.
\renewcommand{\kall}[3][black]{\mall{#1}{#2}{#3}}

% Environment "kdefinition" has effect only in poster style, thus in math style may be safely deleted.

%Continuation of a syntax definition on a new line
\newcommand{\syntaxContNewLine}[3][\defSort]{\par\indent\rulebox{%
  $\setlength{\syntaxlength}{\widthof{$\mathrel{::=}$}}%
  \setlength{\syntaxlength}{.5\syntaxlength}%
  \addtolength{\syntaxlength}{\widthof{\syntaxKeyword$#1$}}%
  \hspace{\syntaxlength}%
  \;\;\;\;\;\;\;{#2}$ \ifthenelse{\equal{#3}{}}{}{[#3]}%
  }%\k@markPosition%
}

%Should be put after a syntaxLong.
\newcommand{\syntaxEnd}[3][\defSort]{
  \indent\rulebox{%
  $\setlength{\syntaxlength}{\widthof{$\mathrel{::=}$}}%
  \setlength{\syntaxlength}{.5\syntaxlength}%
  \addtolength{\syntaxlength}{\widthof{\syntaxKeyword$#1$}}%
  \hspace{\syntaxlength}$%
  }%\k@markPosition%
}

\newcommand{\syntaxLong}[3][\defSort]{\rulebox{%
\syntaxKeyword
$
  \begin{array}[t]{@{}l@{}}
  #1 \\
  \mathrel{::=}{#2}
  \end{array}
$ {}%
}%\k@markPosition%
}

% Grigore's idea macro
\newcommand{\idea}[1]{
  \begin{quote}
    \rule{.45\textwidth}{.5pt}\newline
    {\em #1}
    \vspace*{-1ex}\newline \rule{.45\textwidth}{.5pt}
  \end{quote}
}

\newenvironment{ideas}
{ \begin{quote}
    \rule{.45\textwidth}{.5pt}
    \newline
    \begin{em}
} {
    \end{em}
    \vspace*{-1ex}
    \leavevmode
    \newline
    \rule{.45\textwidth}{.5pt}
  \end{quote}
}

%Enforcing black cells
\renewcommand{\kall}[3][white]{\mall{black}{#2}{#3}}
\renewcommand{\kallLarge}[3][white]{\mallLarge{black}{#2}{#3}}
\renewcommand{\kprefix}[3][white]{\mprefix{black}{#2}{#3}}
\renewcommand{\ksuffix}[3][white]{\msuffix{black}{#2}{#3}}
\renewcommand{\kmiddle}[3][white]{\mmiddle{black}{#2}{#3}}

% Settigns required for Chucky's background section
\usepackage{acronym}

\providecommand{\Sec}{}
\renewcommand{\Sec}{Section~}
\newcommand{\Fig}{Figure~}

\newcommand{\cellname}[1]{\textsf{#1}}

\newcommand{\kequation}[2]{\begin{equation*}{\small#2}\end{equation*}}

%Probably a mapsto with spacing
\newcommand{\mapstox}{\small\mathrel{\mapsto}}

%For spacing between cell lines
%\newcommand{\kBR}{\\[0.3em]}
