\section{Introduction}
Distributed applications are all pervasive, and have been studied at great depth. Development of distributed applications with physical control components have gained great importance in recent history due to the ever increasing automation in everyday life. These applications can run on platforms which interact with the physical environment (UAVs, self-driven cars, etc.). Development of such applications usually requires expertise of the language in which the applications are written, and experience with the hardware platform on which it will be deployed.  Managing and experimenting with these platforms is usually a time consuming and non trivial task, as most such systems are based on design choices specific to developers abd not standardized.
Deployment and testing of a new algorithm might take a long time for even experienced programmer.  If one wanted to test an existing application different dynamics models, or a new platform, or heterogeneous agents, it could jeopardize time sensitive deployment programs.  Considerably less effort is needed to perform this type of task with desktop (or mobile) applications. The lack of an accepted standard, and abstractions for specific applications makes experiments are costly to repeat and verify. 

In this paper, we present \rolang: a language which enables users to start developing and understanding applications which interact with the physical environment. The user writes code for a single agent, which can be deployed on a multiple agents to perform a distributed task, such as leader election . \rolang provides a clean precondition-effect style of programming, similar to the representation of Hybrid Input-Output Automata as in~\cite{lynch1996hybrid}. HIOA is a mathematical framework for precisely describing
systems that have both discrete and continuous behaviors. The precondition-effect block constitute \emph{events}, which correspond to discrete steps in the HIOA. Control variables change when the external functions are called, corresponding to the continuous trajectory of the HIOA, and this take non-zero time to occur. We introduce a time model in \refsect{time} which guards against zeno behavior of the applications.  

 We designed the semantics of this language, driven by one simple idea. The effect of the interaction with the physical environment is seen only on "control" variables, or variables which are controlled outside the environment. Provided controllers to determine the values of these variables when looked up, the semantics of this language can be specified independently, considering these controllers as external functions which return  values for these variables. We provide several abstractions to the user for communication and physical control. The major design details are captured by the semantics we present in this paper, while the external functions can be implemented by the user if they want to do so. In applications written in \rolang, agents communicate using shared variables. Shared variables in practice are implemented through message passing on hardware platforms. In this paper, we present a semantics which has an eventual consistency, but our design is modular enough to implement different consistency models. In our semantics, the external functions which control interaction with the physical environment are \emph{parameters} provided to the semantics. Several other application specific features can also be provided as external functions, such as communication restrictions based on geographical proximity. We assume that all agents can communicate with all other agents. In \refsect{future}, we discuss our planned implementation of a publish-subscribe model of communication among agents. 

We use the \K semantic framework to write the semantics of this language, as it can be used to generate an executable semantics which lets us run applications and generate execution traces. We talk about \K in \refsect{k}. We implemented several distributed applications using this language, and we discuss two of them in this paper.The executable semantics generated by the \K framework can be used to run applications written in this language.\footnote{https://github.com/ritwika314/PCCL} 