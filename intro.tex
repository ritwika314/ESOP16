\section{Introduction}
 Distributed applications that monitor and control the physical environment have gained  prominence with the rise of 
 supply chain management~\cite{kivaFORBES,ref:Wurman08}, factory automation~\cite{dilts1991evolution,ref:Correll08a,greenfield2003software}, and Internet of Things (IoT).
 While new capabilities and systems are deployed routinely,  testing, debugging, and formal verification remain monumental~\cite{geels2006replay,mclurkin2006speaking}.
 %
 Part of the challenge arises from the fact that the current language abstractions~\cite{quigley2009ros,gerkey2003player,blank2003python,corke1996robotics,nesnas2007claraty,nesnas2007claraty2,calisi2008openrdk} are inherited from those that are used for developing standalone applications and  do not provide any abstractions for programming open distributed systems that monitor and control the physical environment. 

Aiming to gain insight about some of these challenges, we are developing a programming system for distributed heterogeneous robotics at Illinois~\cite{ZimMit:2012}. In this paper, we present the design of a language \rolang and its formal semantics. The language provides abstractions for distributed computational nodes or agents to interact with each other and with the physical environment. The user writes code for a single agent, which can be deployed on a multiple (possibly heterogeneous) agents to perform a distributed tasks, such as leader election, formation, distributed search, and distributed SLAM~\cite{cunningham2010ddf}. \rolang provides a  precondition-effect style of programming, 

 Control variables change when the external functions are called, corresponding to the continuous trajectory of the HIOA, and this take non-zero time to occur. We introduce a time model in \refsect{time} which guards against zeno behavior of the applications.  

 We designed the semantics of this language, driven by one simple idea. The effect of the interaction with the physical environment is seen only on "control" variables, or variables which are controlled outside the environment. Provided controllers to determine the values of these variables when looked up, the semantics of this language can be specified independently, considering these controllers as external functions which return  values for these variables. We provide several abstractions to the user for communication and physical control. The major design details are captured by the semantics we present in this paper, while the external functions can be implemented by the user if they want to do so. In applications written in \rolang, agents communicate using shared variables. Shared variables in practice are implemented through message passing on hardware platforms. In this paper, we present a semantics which has an eventual consistency, but our design is modular enough to implement different consistency models. In our semantics, the external functions which control interaction with the physical environment are \emph{parameters} provided to the semantics. Several other application specific features can also be provided as external functions, such as communication restrictions based on geographical proximity. We assume that all agents can communicate with all other agents. In \refsect{future}, we discuss our planned implementation of a publish-subscribe model of communication among agents. 

We use the \K semantic framework to write the semantics of this language, as it can be used to generate an executable semantics which lets us run applications and generate execution traces. We talk about \K in \refsect{k}. We implemented several distributed applications using this language, and we discuss two of them in this paper.The executable semantics generated by the \K framework can be used to run applications written in this language.\footnote{https://github.com/ritwika314/PCCL} 